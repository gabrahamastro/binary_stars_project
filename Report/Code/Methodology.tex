\section{Methodology}\label{sec:Method}

We assessed the efficacy of TESS and Gaia DR3 photometry in identifying benchmarks for observation by PLATO. We performed a two-stage survey using \verb|Lightkurve| on a catalogue of EBs from \citet{Prsa22}.\\

EBs with morphologies $< 0.6$ were pre-selected to exclude ellipsoidal variables. In the first stage, we examined the TESS light curves and selected EBs with high signal-to-noise ratios, narrow eclipses and minimal distortion. Orbital periods were estimated by phase-folding the light curves. For systems with long-period variations, we detrended the data by dividing the light curves by fitted polynomials. These detrended light curves were then processed using \verb|jktebop| (task 3), and model light curves with residuals $< 0.02$ were generated.\\

In the second stage, we compared the TESS light curves with corresponding Gaia data, selecting EBs with primary and secondary eclipses in the $G-$, $BP-$ and $RP-$bands. We used the parameters from the TESS analysis to fit Gaia model light curves, accounting for differences in reference time. Limb darkening coefficients ($h1,\ h2$) for the Gaia data were calculated, assuming effective temperatures $5000-8000 \mathrm{\ K}$, zero metallicity, microturbulent velocity $2 \mathrm{\ km \ s^{-1}}$, and logarithmic surface gravity $4 \mathrm{\ cm\ s^{-2}}$. These values were used to generate the power-2 limb darkening parameters interpolated from \citet{Claret22}. $h1$ and $h2$ values were computed using equations (14) and (15) from \citet{Southworth23}. Uncertainties in the Gaia model parameters were derived using Monte Carlo simulations (task 8, \verb|jktebop|). \\

The output scale factors and flux ratios from task 8 were used to estimate the primary and secondary apparent magnitudes, from which the corresponding colour indices ($BP-RP$) and $G-$band absolute magnitudes were calculated. Parallax, extinction, and colour excess values were obtained from \citet{Gaia23}. Systems with limited data were assumed to have zero extinction and colour excess. Finally, we estimated stellar masses by interpolating our results with \citet{Mamajek22}. Our ten best-characterised EBs are presented on a CMD (Figure \ref{fig:CMD}).
